%%% Copyright (C) 2018 Vincent Goulet
%%%
%%% Ce fichier fait partie du projet
%%% «Théorie de la crédibilité avec R»
%%% http://github.com/vigou3/theorie-credibilite-avec-r
%%%
%%% Cette création est mise à disposition selon le contrat
%%% Attribution-Partage dans les mêmes conditions 4.0
%%% International de Creative Commons.
%%% http://creativecommons.org/licenses/by-sa/4.0/

\chapter{Crédibilité de stabilité}
\label{chap:stabilite}

La théorie de la crédibilité est apparue dans le domaine des accidents
du travail au début des années 1900. Les actuaires cherchaient alors à
déterminer la taille minimale (en nombre d'employés) qu'un employeur
devait atteindre pour que son expérience soit considérée pleinement
«fiable» (\emph{dependable}) ou «crédible».

Tel que relaté à la \autoref{sec:introduction-historique:historique},
\cite{Mowbray:1914} fournit une première solution. Il fournit du même
coup une définition de ce que doit être une prime pure crédible: «une
prime pour laquelle la probabilité est forte qu'elle ne diffère pas de
la vraie prime par plus d'une limite arbitraire».

En termes mathématiques, nous souhaitons que
\begin{displaymath}
  \Pr[(1 - k) \esp{S} \leq S \leq (1 + k) \esp{S}] \geq p,
\end{displaymath}
où $S$ représente l'expérience d'un contrat, sous une forme ou une
autre; $k$ est une petite valeur, habituellement autour de $5$~\%; $p$
une valeur près de $1$, habituellement $0,90$, $0,95$ ou $0,99$.


\section{Crédibilité complète}
\label{sec:stabilite:complete}

En crédibilité complète, un contrat d'assurance est considéré
\emph{crédible} si son expérience est \emph{stable}. Lorsque c'est le
cas, seule l'expérience du contrat est utilisée dans la tarification
de ce contrat.

\tipbox{La notion de portefeuille de contrats ne joue aucun rôle pour
  le moment.}

Intuitivement, la stabilité de l'expérience d'un contrat va de pair
avec sa «taille», qu'elle soit exprimée en termes de: volume de prime,
masse salariale, nombre d'employés, nombre de sinistres, nombre
d'années d'expérience, etc.

De plus, la taille du contrat est généralement liée à la fréquence des
sinistres, et non à la sévérité de ceux-ci.

\begin{definition}[Crédibilité complète]
  \label{def:stabilite:complete}
  Une crédibilité complète d'ordre $(k, p)$ est attribuée à
  l'expérience $S$ d'un contrat si les paramètres de la distribution
  de $S$ sont tels que la relation
  \begin{displaymath}
    \Pr[(1 - k) \esp{S} \leq S \leq (1 + k) \esp{S}] \geq p
  \end{displaymath}
  est vérifiée.
\end{definition}

La variable aléatoire $S$ représente la somme de l'expérience de
plusieurs risques. Par le Théorème central limite,
\begin{equation*}
  Z = \frac{S - \esp{S}}{\sqrt{\var{S}}}
  \longrightarrow
  N(0, 1)
\end{equation*}
lorsque le nombre de termes dans la somme $S$ augmente. La relation de
la \autoref{def:stabilite:complete} peut alors se réécrire
\begin{align*}
  \Pr
  \left[
    - \frac{k \esp{S}}{\sqrt{\var{S}}}
    \leq Z \leq
    \frac{k \esp{S}}{\sqrt{\var{S}}}
  \right]
  &\approx
    \Phi
    \left(
      \frac{k \esp{S}}{\sqrt{\var{S}}}
    \right) -
    \Phi
    \left(
      - \frac{k \esp{S}}{\sqrt{\var{S}}}
    \right) \\
  &=
    2 \Phi
    \left(
    \frac{k \esp{S}}{\sqrt{\var{S}}}
    \right) - 1 \geq p.
\end{align*}
L'inégalité est satisfaite dès lors que
\begin{equation*}
  \frac{k \esp{S}}{\sqrt{\var{S}}} \geq \zeta_{\varepsilon/2},
\end{equation*}
ou, de manière équivalente, lorsque
\begin{equation}
  \label{eq:stabilite:complete}
  \esp{S} \geq
  \left(
    \frac{\zeta_{\varepsilon/2}}{k}
  \right)
  \sqrt{\var{S}},
\end{equation}
où $\varepsilon = 1 - p$ et $\zeta_\alpha$ est le $100(1 -
\alpha)${\ieme} centile d'une loi normale centrée réduite.

\begin{exemple}[Binomiale pure]
  \label{exemple:stabilite:binomiale_pure}
  \citet{Mowbray:1914} recherchait le nombre minimal d'employés pour
  considérer l'expérience d'un employeur pleinement crédible. Ses
  hypothèses étaient que la variable aléatoire $S$ du nombre
  d'accidents par année pour un employeur est binomiale de paramètres
  $n$ et $\theta$, où $n$ représente le nombre d'employés et $\theta$
  la probabilité qu'un accident survienne au cours de l'année pour
  tout employé. Cette probabilité est supposée connue.

  Avec $\esp{S} = n \theta$ et $\var{S} = n \theta (1 - \theta)$ et en
  isolant $n$ dans l'inégalité \eqref{eq:stabilite:complete}, nous
  obtenons la relation suivante pour le critère de crédibilité
  complète:
  \begin{displaymath}
    n \geq
    \left(
      \frac{\zeta_{\varepsilon/2}}{k}
    \right)^2
    \frac{1 - \theta}{\theta}.
  \end{displaymath}
  \qed
\end{exemple}

\begin{exemple}[Poisson composée]
  \label{exemple:stabilite:comppois}
  La taille d'un contrat est souvent exprimée en termes du nombre
  espéré de sinistres dans une période --- typiquement une année. La
  distribution la plus populaire pour $S$ dans un tel cas est alors la
  Poisson composée, c'est-à-dire
  \begin{displaymath}
    S = X_1 + \dots + X_N
  \end{displaymath}
  où $N \sim \text{Poisson}(\lambda)$ et la distribution de $X_1, X_2,
  \dots$ est $F_X(\cdot)$. Nous savons que
  \begin{align*}
    \esp{S}
    &= \esp{N} \esp{X} \\
    &= \lambda \esp{X} \\
    \var{S}
    &= \var{N} \esp{X}^2 + \esp{N} \var{X} \\
    &= \lambda \esp{X}^2 + \lambda \var{X} \\
    &= \lambda \esp{X^2}.
  \end{align*}
  La relation \eqref{eq:stabilite:complete} devient alors, en isolant
  le nombre espéré de sinistre $\lambda$:
  \begin{equation}
    \label{eq:stabilite:complete_comppois}
    \begin{split}
      \lambda
      &\geq
      \left(
        \frac{\zeta_{\varepsilon/2}}{k}
      \right)^2
      \left(
        1 + \frac{\var{X}}{\esp{X}^2}
      \right) \\
      &=
      \left(
        \frac{\zeta_{\varepsilon/2}}{k}
      \right)^2
      (1 + \CV(X)^2),
    \end{split}
  \end{equation}
  où
  \begin{displaymath}
    \CV(X) = \frac{\sqrt{\var{X}}}{\esp{X}}
  \end{displaymath}
  est le coefficient de variation de la variable aléatoire $X$.

  Vous remarquerez que, dans l'inégalité
  \eqref{eq:stabilite:complete_comppois}, le seuil de crédibilité
  complète $\lambda$ augmente avec la variabilité des montants de
  sinistres (exprimée en fonction de coefficient de variation
  $\CV(X)$).

  Si $k = 0,05$ et $p = 0,90$, alors $\zeta_{0,05} = 1,645$ et
  \begin{displaymath}
    \lambda
    \geq
    \nombre{1082,41}
    \left(
      1 + \frac{\var{X}}{\esp{X}^2}
    \right).
  \end{displaymath}

  Si, de plus, $X$ est une variable aléatoire dégénérée (c'est-à-dire
  que $\Pr[X = M] = 1$ pour $M$ quelconque), alors
  \begin{displaymath}
    \lambda \geq \nombre{1082,41}.
  \end{displaymath}
  Ce cas revient en définitive à poser
  $S \sim \text{Poisson}(\lambda)$. %
  \qed
\end{exemple}

\tipbox{Le seuil de pleine crédibilité \nombre{1082} est un nombre
  célèbre en théorie de la crédibilité. Malheureusement, il est très
  souvent utilisé à tort et à travers et sans aucunement tenir compte
  des hypothèses qui nous ont permis d'obtenir cette valeur.}

\begin{exemple}
  \label{exemple:stabilite:nombre_annees}
  Dans les deux exemples précédents, nous avons déterminé le seuil de
  pleine crédibilité d'un contrat pour une seule période. Nous
  pourrions aussi choisir de fixer le seuil en fonction du nombre
  d'années d'expérience. Pour cela, il suffit de définir
  \begin{displaymath}
    W = \frac{S_1 + \dots + S_n}{n},
  \end{displaymath}
  où $S_1, \dots, S_n$ sont des variables aléatoires indépendantes et
  identiquement distribuées et $S_t$ est l'expérience de l'année
  $t = 1, \dots, n$.  Nous avons alors
  \begin{align*}
    \esp{W} &= \esp{S_t} \\
    \var{W} &= \frac{\var{S_t}}{n}.
  \end{align*}

  En appliquant le critère de crédibilité complète sur la variable
  aléatoire $W$, nous obtenons une expression en tous points
  équivalente à l'inégalité \eqref{eq:stabilite:complete} dans
  laquelle nous remplaçons l'espérance et la variance par les
  expressions ci-dessus. L'expérience d'un contrat est considérée
  pleinement crédible après
  \begin{equation*}
    n \geq
    \left(
      \frac{\zeta_{\varepsilon/2}}{k}
    \right)^2
    \frac{\var{S_t}}{\esp{S_t}^2}
  \end{equation*}
  périodes.
  \qed
\end{exemple}

\tipbox{La distribution de $S$ n'est en général pas symétrique. Est-il
  alors correct d'utiliser le Théorème central limite (TCL)? Le
  critère de crédibilité complète exige que la distribution de $S$
  soit très concentrée autour de sa moyenne, donc presque symétrique.
  Le TCL s'avère donc bien assez précis.}

\importantbox{Il y a en fait très peu d'applications légitimes de la
  crédibilité de stabilité. Un bon exemple, toutefois: la fixation du
  seuil d'admissibilité à un régime de tarification rétrospectif, où
  la stabilité de l'expérience joue un rôle important.}


\section{Crédibilité partielle}
\label{sec:stabilite:partielle}

La crédibilité complète ne connait que deux états: aucune crédibilité
ou crédibilité complète. Le besoin (ou la pression) de tenir compte en
partie de l'expérience individuelle d'un contrat se trouvant sous le
seuil de crédibilité complète apparait rapidement chez les assureurs.

\citet{Whitney:1918} propose de pondérer l'expérience individuelle
($S$) et la prime collective ($m$) par un \emph{facteur de
  crédibilité} ($0 \leq z \leq 1$) en une prime de la forme
\begin{displaymath}
  \pi = z S + (1 - z) m.
\end{displaymath}

Plusieurs formules différentes ont été proposées pour $z$. Soit $n_0$
le seuil de crédibilité complète. Les formules les plus populaires
sont
\begin{align*}
  z &= \min \left\{ \sqrt{\frac{n}{n_0}}, 1 \right\}, \\
  z &= \min \left\{ \left( \frac{n}{n_0} \right)^{2/3}, 1 \right\}, \\
  \intertext{et la formule de Whitney}
  z &= \frac{n}{n + K},
\end{align*}
où $K$ est une constante déterminée au jugement de façon à limiter les
fluctuations dans la prime d'une année à l'autre (\emph{swing}).

Le but de cette approche consiste à incorporer autant d'expérience
individuelle que possible dans la prime sans trop en affecter la
stabilité d'une année à l'autre.

\importantbox{La distribution des primes dans cette approche de
  crédibilité partielle est basée uniquement sur la taille des
  assurés. Rien n'assure que la tarification est précise ou
  équitable.}


\section{Exercices}
\label{sec:stabilite:exercices}

\Opensolutionfile{reponses}[reponses-stabilite]
\Opensolutionfile{solutions}[solutions-stabilite]

\begin{Filesave}{solutions}
\section*{Chapitre \ref*{chap:stabilite}}
\addcontentsline{toc}{section}{Chapitre \protect\ref*{chap:stabilite}}

\end{Filesave}

\begin{exercice}
  Le montant total des sinistres, $S$, a une distribution binomiale
  composée de paramètres $n = \nombre{1000}$ et $\theta = 0,6$. La
  distribution du montant des sinistres est une log-normale de
  paramètres $\mu = 3$ et $\sigma² = 4$. Calculer $\esp{S}$ et
  $\var{S}$.
  \begin{rep}
    $\esp{S} = \nombre{89047}$, $\var{S} = \nombre{713633042}$
  \end{rep}
  \begin{sol}
    Nous avons $S = X_1, \dots, X_N$ avec
    $N \sim \text{Binomiale}(\nombre{1000},\, 0,6)$ et
    $X \sim \text{Log-normale}(3, 4)$, d'où
    \begin{align*}
      \esp{N} &= (\nombre{1000})(0,6) = 600 \\
      \var{N} &= (\nombre{1000})(0,6)(1 - 0,6) = 240 \\
      \esp{X} &= e^{3 + 4/2} = e^5 \\
      \var{X} &= e^{2 \cdot 3 + 4}(e^4 - 1) = e^{10}(e^4 - 1) \\
      \intertext{et, par conséquent,}
      \esp{S}
      &= \esp{N} \esp{X} \\
      &= 600\, e^5 \\
      &= \nombre{89047} \\
      \var{S}
      &= \esp{N} \var{X} + \var{N} \esp{X}^2 \\
      &= 600\, e^{10}(e^4 - 1) + 240\, e^{10} \\
      &= \nombre{713633042}.
    \end{align*}
  \end{sol}
\end{exercice}

\begin{exercice}
  Sachant que le nombre de sinistres a une distribution binomiale
  négative de paramètres $r = 4$ et $\theta = 0,5$ et que la sévérité
  d'un sinistre a une distribution gamma de paramètres $\alpha = 2$ et
  $\lambda = 0,5$ calculer l'espérance et la variance du montant total
  des sinistres, $S$.
  \begin{rep}
    $\esp{S} = 16$, $\var{S} = 160$
  \end{rep}
  \begin{sol}
    Nous avons $\esp{S} = \esp{N}\esp{X}$ et $\var{S} = \esp{N}\var{X} +
    \esp{X}^2\var{N}$. Or, $\esp{N} = 4 (0,5)/(1 - 0,5) = 4$, $\var{N}
    = 4 (0,5)/(1 - 0,5)^2 = 8$, $\esp{X} = 2/0,5 = 4$ et $\var{X} =
    2/0,5^2 = 8$, d'où
    \begin{align*}
        \esp{S}
        &= (4)(4)
         = 16 \\
        \intertext{et}
        \var{S}
        &= (4)(8) + (4^2)(8)
         = 160.
    \end{align*}
  \end{sol}
\end{exercice}

\begin{exercice}
  Interpréter l'inégalité suivante:
  \begin{displaymath}
    \Pr
    \left[
      0,96 \esp{S} \leq S \leq 1,04 \esp{S}
    \right]
    \geq 0,95.
  \end{displaymath}
  \begin{rep}
    L'inégalité signifie que l'on veut être certain à au moins $95$~\%
    que l'expérience individuelle d'un assuré ne varie pas de plus de
    $4$~\% autour de la prime pure.
  \end{rep}
\end{exercice}

\begin{exercice}
  \label{ex:stabilite:nPC}
  Dériver une formule générale pour déterminer le niveau de
  crédibilité complète d'ordre $(k, p)$ pour $\bar{S} = (S_1 + \dots +
  S_n)/n$ lorsque la distribution du montant total des sinistres est
  une Poisson composée.
  \begin{rep}
    $n \geq \lambda^{-1} (\zeta_{\varepsilon/2}/k)^2 (1 + \CV(X)^2)$
  \end{rep}
  \begin{sol}
    Puisque $\esp{\bar{S}} = \esp{S}$ et $\var{\bar{S}} = \var{S}/n$,
    l'inégalité
    \begin{displaymath}
      \Pr
      \left[
        (1 - k) \esp{\bar{S}} \leq \bar{S} \leq (1 + k) \esp{\bar{S}}
      \right]
      \geq p
    \end{displaymath}
    est satisfaite lorsque
    \begin{displaymath}
      n \ge
      \left( \frac{\zeta_{\varepsilon/2}}{k} \right)^2
      \frac{\var{S}}{\esp{S}^2}.
    \end{displaymath}
    Nous savons que dans le cas d'une Poisson composée,
    $\esp{S} = \lambda \esp{X}$ et
    $\var{S} = \lambda (\var{X} + \esp{X}^2)$. Ainsi,
    \begin{align*}
      n
      & \geq \left( \frac{\zeta_{\varepsilon/2}}{k} \right)^2
      \frac{\var{X} + \esp{X}^2}{\lambda \esp{X}^2} \\
      &= \frac{1}{\lambda} \left( \frac{\zeta_{\varepsilon/2}}{k}
      \right)^2 \left( 1 + \frac{\var{X}}{\esp{X}^2}
      \right) \\
      &= \frac{1}{\lambda} \left( \frac{\zeta_{\varepsilon/2}}{k}
      \right)^2 (1 + \CV(X)^2),
    \end{align*}
    où $\CV(X)$ est le coefficient de variation de $X$.
  \end{sol}
\end{exercice}

\begin{exercice}
  Soit $N \sim \text{Poisson}(256)$, $X \sim \text{Pareto}(3,\, 0,05)$
  et $S = X_1 + \dots + X_N$.
  \begin{enumerate}
  \item Quelle est la plus petite marge d'erreur admissible autour de
    $\esp{S}$ faisant toujours en sorte que $S$ a une crédibilité
    complète à 90~\%?  Interpréter brièvement le résultat.
  \item Quelle est la plus petite marge d'erreur admissible autour de
    $\esp{\bar{S}}$ faisant toujours en sorte que $\bar{S}$ a une
    crédibilité complète à 90~\% après 10 années?
  \end{enumerate}
  \begin{rep}
    \begin{inparaenum}
    \item $k \geq 0,2056$
    \item $k \geq 0,065$
    \end{inparaenum}
  \end{rep}
  \begin{sol}
    \begin{enumerate}
    \item Nous savons de l'\autoref{exemple:stabilite:comppois} que
      dans le cas d'une Poisson composée, la crédibilité complète est
      donnée par
      \begin{displaymath}
        \lambda \geq
        \left(
          \frac{\zeta_{\varepsilon/2}}{k}
        \right)^2
        \left(
          1 + \frac{\var{X}}{\esp{X}^2}
        \right).
      \end{displaymath}
      Or, nous avons $\esp{X} = \theta/(\alpha - 1) = 1/40$ et
      $\var{X} = \alpha \theta^2/((\alpha - 1)^2(\alpha - 2)) =
      3/\nombre{1600}$. Comme nous exigeons un niveau de crédibilité
      de $90$~\%, $\zeta_{\varepsilon/2} = 1,645$. En isolant $k$ dans
      la formule, nous obtenons
      \begin{displaymath}
        k \ge
        \frac{1,645}{\sqrt{256}}
        \sqrt{1 + \frac{3/\nombre{1600}}{1/\nombre{1600}}} =
        0,2056 = 20,56~\%.
      \end{displaymath}
      La valeur de $k$ est obtient une grande parce que la
      distribution du montant des sinistres est très asymétrique. Pour
      obtenir une valeur de $k$ plus usuelle (de l'ordre de $5$~\%),
      il faudrait augmenter le paramètre de Poisson, de manière à
      sommer un plus grand nombre de lois de Pareto.
    \item De manière similaire, mais en utilisant plutôt la formule
      \begin{displaymath}
        n \geq \frac{1}{\lambda}
        \left(
          \frac{\zeta_{\varepsilon/2}}{k}
        \right)^2
        \left(
          1 + \frac{\var{X}}{\esp{X}^2}
        \right)
      \end{displaymath}
      obtenue à l'\autoref{ex:stabilite:nPC}
      avec $n = 10$, nous obtenons
      \begin{displaymath}
        k \geq
        \frac{1}{\sqrt{256}} \frac{1,645}{\sqrt{10}}
        \sqrt{1 + \frac{3/\nombre{1600}}{1/\nombre{1600}}} = 0,065 = 6,5~\%.
      \end{displaymath}
      La marge d'erreur est plus petite qu'en a) puisque la
      distribution de $\bar{S}$ après 10 années est le résultat de la
      somme d'un beaucoup plus grand nombre de sinistres.
    \end{enumerate}
  \end{sol}
\end{exercice}

\begin{exercice}
  Sachant que la variable aléatoire $S$ de l'expérience des contrats
  obéit à une loi $N(\mu, \sigma^2)$, trouver, pour une période
  d'expérience, la relation entre $\mu$ et $\sigma$ pour avoir une
  crédibilité complète d'ordre $(k, p)$ pour chacune des combinaisons
  de $k$ et $p$ suivantes.
  \begin{enumerate}
  \item $(0,04, 0,95)$
  \item $(0,05, 0,90)$
  \item $(0,01, 0,98)$
  \end{enumerate}
  \begin{rep}
    \begin{inparaenum}
    \item $\mu \geq 49 \sigma$
    \item $\mu \geq 32,9 \sigma$
    \item $\mu \geq 232,6 \sigma$
    \end{inparaenum}
  \end{rep}
  \begin{sol}
    Dans un tel cas normale pure, la formule générale
    \eqref{eq:stabilite:complete} devient simplement
    \begin{displaymath}
      \mu \geq \left( \frac{\zeta_{\varepsilon/2}}{k} \right)
      \sigma.
    \end{displaymath}
    \begin{enumerate}
    \item On a $k = 0,04$ et $p = 0,95$ d'où $\zeta_{0,025} = 1,96$ et
      donc $\mu \geq 49 \sigma$.
    \item On a $k = 0,05$ et $p = 0,90$ d'où $\zeta_{0,05} = 1,645$ et
      donc $\mu \geq 32,9 \sigma$.
    \item On a $k = 0,01$ et $p = 0,98$ d'où $\zeta_{0,01} = 2,326$ et
      donc $\mu \geq 232,6 \sigma$.
    \end{enumerate}
  \end{sol}
\end{exercice}

\begin{exercice}
  Sachant que $S \sim N(\mu, \sigma^2)$ et $\bar{S} = (S_1 + \dots +
  S_9)/9$, où $S_1, \dots, S_9$ sont toutes des variables aléatoires
  mutuellement indépendantes distribuées comme $S$, déterminer la
  relation entre $\mu$ et $\sigma$ faisant en sorte que $\bar{S}$ a
  une crédibilité complète d'ordre $(k, p)$ pour chacune des
  combinaisons de $k$ et $p$ suivantes.
  \begin{enumerate}
  \item (0,05, 0,90)
  \item (0,05, 0,95)
  \item (0,01, 0,90)
  \item (0,01, 0,95)
  \end{enumerate}
  \begin{rep}
    \begin{inparaenum}
    \item $\mu \geq 10,97 \sigma$
    \item $\mu \geq 13,07 \sigma$
    \item $\mu \geq 54,83 \sigma$
    \item $\mu \geq 65,33 \sigma$
    \end{inparaenum}
  \end{rep}
  \begin{sol}
    On nous donne que $\bar{S} \sim N(\mu, \sigma^2/9)$. La
    crédibilité complète d'ordre $(k, p)$ est donc atteinte lorsque
    \begin{displaymath}
      \esp{\bar{S}} \geq \left( \frac{\zeta_{\varepsilon/2}}{k} \right)
      \sqrt{\var{\bar{S}}},
    \end{displaymath}
    soit
    \begin{displaymath}
      \mu \geq \left( \frac{\zeta_{\varepsilon/2}}{k} \right)
      \frac{\sigma}{3}.
    \end{displaymath}
    \begin{enumerate}
    \item On a $k = 0,05$ et $p = 0,90$ d'où $\zeta_{0,05} = 1,645$ et
      $\mu \geq 10,97 \sigma$.
    \item On a $k = 0,05$ et $p = 0,95$ d'où $\zeta_{0,025} = 1,96$ et
      $\mu \geq 13,07 \sigma$.
    \item On a $k = 0,01$ et $p = 0,90$ d'où $\zeta_{0,05} = 1,645$ et
      $\mu \geq 54,83 \sigma$.
    \item On a $k = 0,01$ et $p = 0,95$ d'où $\zeta_{0,025} = 1,96$ et
      $\mu \geq 65,33 \sigma$.
    \end{enumerate}
  \end{sol}
\end{exercice}

\begin{exercice}
  \label{ex:stabilite:rBNC}
  Juliette travaille au département de tarification d'une compagnie
  d'assurance très active en assurance automobile. À la suite d'une
  analyse de données exhaustive, Juliette peut affirmer que la
  fréquence des sinistres pour ce type de produit a une distribution
  binomiale négative de paramètres $r$ et $\theta = 0,01$. La sévérité
  des sinistres, quant à elle, suit une loi gamma de paramètres
  $\alpha = 0,02$ et $\lambda = 1$. Trouver la plus petite valeur de
  $r$ telle que le montant total des sinistres d'un contrat
  d'assurance automobile sera à plus ou moins $5$~\% égal à sa moyenne,
  $19$ fois sur $20$.
  \begin{rep}
    $r \geq \nombre{2328,24}$
  \end{rep}
  \begin{sol}
    Déterminons tout d'abord une formule générale pour le cas de la
    binomiale négative composée. Sachant que
    $\esp{N} = r(1 - \theta)/\theta$ et que
    $\var{N} = r (1 - \theta)/\theta^2$, nous obtenons
    \begin{displaymath}
      r \geq
      \left(
        \frac{\zeta_{\varepsilon/2}}{k}
      \right)^2
      \left(
        \frac{1}{1 - \theta} +
        \frac{\theta}{1-\theta} \frac{\var{X}}{\esp{X}^2}
      \right).
    \end{displaymath}
    Or, dans cet exercice, $\theta = 0,01$, $\var{X}/\esp{X}^2 =
    1/\alpha = 50$ et $\zeta_{\varepsilon/2}/k = 1,96/0,05 = 39,2$. Par
    conséquent, $r \geq \nombre{2328,24}$.
  \end{sol}
\end{exercice}

\begin{exercice}
  Soit $S_j$ le montant total des sinistres pour un assuré à la
  période $j = 1, \dots, n$ tel que
  \begin{displaymath}
    S_j = X_1 + X_2 + \dots + X_{N_j},
  \end{displaymath}
  où $X_1, X_2, \dots$ sont les montants individuels des sinistres
  dont la distribution est dégénérée en $M$ et $N_j$ suit une loi de
  Poisson de paramètre $\lambda$. Calculer le nombre total de
  sinistres espéré minimal pour accorder une crédibilité complète
  d'ordre $(0,04, 0,90)$ à l'expérience individuelle $\bar{S}$.
  \begin{rep}
    \nombre{1692}
  \end{rep}
  \begin{sol}
    Remarquez tout d'abord que le contexte est celui de
    l'\autoref{ex:stabilite:nPC}, sauf que le critère de crédibilité
    complète est exprimé non pas en nombre d'années d'expérience, $n$,
    mais plutôt en nombre total de sinistres espéré, soit $n \lambda$.
    Nous avons donc
    \begin{displaymath}
      n \lambda \geq
      \left(
        \frac{\zeta_{\varepsilon/2}}{k}
      \right)^2
      (1 + \CV(X)^2).
    \end{displaymath}
    Or, dans le cas présent, $\CV(X) = 0$ puisque $\Pr[X = M]
    = 1$ (distribution dégénérée en $M$), $k = 0,04$ et $\zeta_{0,05}
    = 1,645$, d'où $n \lambda \geq \nombre{1692}$.
  \end{sol}
\end{exercice}

\begin{exercice}
  Soit $S_{j}$ le montant total des sinistres pour un assuré à la
  période $j = 1, \dots, n$ et
  $S_{j} \sim \text{Poisson composée}(\lambda, F_X(\cdot))$. Le
  montant d'un sinistre individuel a une variance de $100$ et une
  espérance de $5$. Déterminer le nombre minimal espéré de sinistres
  pour accorder une crédibilité complète selon les critères
  ci-dessous.
  \begin{enumerate}
  \item Le montant total des sinistres demeure à $3$~\% ou moins du
    montant espéré avec une probabilité de $95$~\%.
  \item Le nombre total de sinistres demeure à $3$~\% ou moins du nombre
    espéré avec une probabilité de $95$~\%.
  \end{enumerate}
  \begin{rep}
    \begin{inparaenum}
    \item $\nombre{21343}$ sinistres
    \item $\nombre{4269}$ sinistres
    \end{inparaenum}
  \end{rep}
  \begin{sol}
    \begin{enumerate}
    \item Le montant total des sinistres a une distribution Poisson
      composée avec $\esp{X} = 5$, $\var{X} = 100$, $k = 0,03$ et
      $\zeta_{\varepsilon/2} = \zeta_{0,025} = 1,96$. De la formule
      générale \eqref{eq:stabilite:complete_comppois}, nous obtenons
      directement $\lambda \geq \nombre{21343}$.
    \item La distribution du nombre de sinistres est une Poisson. Nous
      pouvons utiliser la formule
      \eqref{eq:stabilite:complete_comppois} avec $\var{X} = 0$, ce
      qui donne $\lambda \geq \nombre{4269}$.
    \end{enumerate}
  \end{sol}
\end{exercice}

\begin{exercice}
  Une compagnie assure deux groupes ayant la même loi de Poisson pour
  la fréquence de leurs sinistres individuels. Cependant, les
  individus du groupe A ne peuvent avoir que des sinistres de $50$
  alors que les individus du groupe B ont des sinistres obéissant à
  une loi gamma de moyenne $50$. Sachant que l'observation de
  \nombre{1000} sinistres est suffisante pour accorder une crédibilité
  complète au groupe A et que \nombre{3000} sinistres sont nécessaires
  pour accorder une crédibilité complète de même ordre au groupe B,
  calculer les paramètres de la loi gamma en question.
  \begin{rep}
    $\alpha = 1/2$, $\lambda = 1/100$
  \end{rep}
  \begin{sol}
    Soit $X_A$ la variable aléatoire du montant des sinistres du
    groupe A et $X_B$ la variable aléatoire du montant des sinistres
    du groupe B. On nous donne $\esp{X_A} = 50$, $\var{X_A} = 0$ et
    $X_B \sim \text{Gamma}(\alpha, \lambda)$ avec
    $\alpha/\lambda = 50$. De plus, la distribution du montant total
    des sinistres est une Poisson composée, donc le seuil de
    crédibilité complète est donné par \eqref{eq:stabilite:complete_comppois}.

    Pour le groupe A $\lambda_A = \nombre{1000}$ et $\CV(X_A) = 0$,
    d'où $(\zeta_{\varepsilon/2}/k)^2 = \nombre{1000}$. Ce facteur est
    identique pour le groupe B, mais $\lambda_B = \nombre{3000}$ et
    $\CV(X_B) = 1/\sqrt{\alpha}$, d'où
    \begin{displaymath}
      \nombre{3000} = \nombre{1000}
      \left(
        1 + \frac{1}{\alpha}
      \right),
    \end{displaymath}
    soit $\alpha = 1/2$. Enfin, de $\alpha/\lambda = 50$ découle que
    $\lambda = 1/100$.
  \end{sol}
\end{exercice}

\begin{exercice}
  Dans un modèle Poisson composée, on accorde une crédibilité complète
  d'ordre $(0,05, p)$ au nombre total de sinistres si le nombre de
  sinistres observé est supérieur à $\nombre{1000}$. Quel doit être le
  nombre minimal de sinistres pour accorder une crédibilité d'ordre
  $(0,25, p)$ au montant total des sinistres si le montant d'un
  sinistre individuel suit une loi Gamma$(2, 2)$?
  \begin{rep}
    60 sinistres
  \end{rep}
  \begin{sol}
    Deux situations sont décrites:
    \begin{enumerate}[1)]
    \item la crédibilité complète est accordée à la fréquence des
      sinistres seulement;
    \item la crédibilité complète est accordée au montant total des
      sinistres.
    \end{enumerate}
    On nous donne donc d'abord le seuil de crédibilité complète dans
    un modèle avec distribution de Poisson pure (ou Poisson composée
    avec une distribution des montants de sinistres dégénérée en 1),
    puis on nous demande de déterminer le seuil dans un modèle avec
    distribution Poisson composée où la distribution du montant des
    sinistres a un coefficient de variation de $1/\sqrt{2}$. Toujours
    à partir de la formule \eqref{eq:stabilite:complete_comppois},
    nous trouvons que, dans un premier temps,
    \begin{displaymath}
      \nombre{1000} =
      \left(
        \frac{\zeta_{\varepsilon/2}}{0,05}
      \right)^2,
    \end{displaymath}
    soit $\zeta_{\varepsilon/2} = 0,05 \sqrt{\nombre{1000}}$. Dans un
    deuxième temps, nous avons
    \begin{align*}
      \lambda
      &\geq
      \left(
        \frac{\zeta_{\varepsilon/2}}{0,25}
      \right)^2
      \left(
        1 + \frac{1}{2}
      \right) \\
      &= 1,5
      \left(
        \frac{0,05 \sqrt{\nombre{1000}}}{0,25}
      \right)^2 \\
      &= 60.
    \end{align*}
  \end{sol}
\end{exercice}

\begin{exercice}
  Pour ses calculs relatifs à l'admission au régime rétrospectif, la
  CNESST peut n'employer que la fréquence des sinistres, ou encore la
  fréquence ainsi que la sévérité. Dans le premier cas, le nombre
  minimal d'employés pour être admissible au régime rétrospectif est
  de $\nombre{6494}$ en supposant que la probabilité d'avoir un accident
  est de $0,04$.  Sachant que le coefficient de variation de la sévérité
  des sinistres est $2$, que devient le seuil d'admissibilité
  lorsque la sévérité est également prise en compte dans les calculs?
  \begin{rep}
    $\nombre{33552}$
  \end{rep}
  \begin{sol}
    Il faut d'abord identifier le modèle pour la fréquence comme étant
    une $\text{Binomiale}(n, \theta)$, où $n$ est le nombre d'employés
    et $\theta$, la probabilité qu'un employé ait un accident. Nous
    devons étudier deux cas: la binomiale pure et la binomiale
    composée. La formule de crédibilité de stabilité générale est
    \begin{displaymath}
      \esp{S} \geq
      \left(
        \frac{\zeta_{\varepsilon/2}}{k}
      \right) \sqrt{\var{S}}.
    \end{displaymath}
    Si $S \sim \text{Binomiale}(n, \theta)$, alors
    \begin{displaymath}
      n \geq
      \left(
        \frac{\zeta_{\varepsilon/2}}{k}
      \right)^2 \frac{1 - \theta}{\theta}
    \end{displaymath}
    et, si $S \sim \text{Binomiale composée}(n, \theta, F_X(\cdot))$,
    \begin{displaymath}
      n \geq
      \left(
        \frac{\zeta_{\varepsilon/2}}{k}
      \right)^2
      \left(
        \frac{1}{\theta}\, \CV(X)^2 +
        \frac{1 - \theta}{\theta}
      \right).
    \end{displaymath}
    (À noter que cette dernière équation revient à la première si
    $X = 1$ avec probabilité $1$, c'est-à-dire si $\CV(X) = 0$.) Nous
    avons $\theta = 0,04$ et, dans le cas d'une binomiale pure, un
    niveau de crédibilité complète de $n = \nombre{6494}$. Par
    conséquent, $(\zeta_{\varepsilon/2}/k)^2 = 270,5833$. En insérant
    cette valeur dans la formule pour la binomiale composée avec
    $\CV(X) = 2$, nous obtenons un niveau de crédibilité complète de
    $n = \nombre{33552}$ employés.
  \end{sol}
\end{exercice}

\begin{exercice}
  Le montant total des sinistres a une distribution Poisson composée
  où les montants de sinistres individuels proviennent d'une loi de
  Pareto de paramètres $\alpha = 3$ et $\lambda = 100$. Lorsque la
  largeur de l'intervalle de confiance autour de la moyenne est $5$~\%
  de celle-ci, le seuil de crédibilité complète est $\nombre{2500}$. On
  décide de changer la distribution de la fréquence des sinistres pour
  une binomiale négative avec $\theta = 0,5$. Si le seuil de
  crédibilité complète et le niveau de confiance de l'intervalle
  autour de la moyenne demeurent tous deux inchangés, quelle est la
  nouvelle largeur de l'intervalle de confiance?
  \begin{rep}
    $\nombre{13975}$
  \end{rep}
  \begin{sol}
    Avec la distribution Poisson composée, le seuil de crédibilité
    complète est
    \begin{displaymath}
      \lambda \geq
      \left( \frac{\zeta_{\varepsilon/2}}{k} \right)^2
      \left( 1 + \frac{\var{X}}{\esp{X}^2} \right).
    \end{displaymath}
    Or, on nous donne $\esp{X} = 100/(3 - 1) = 50$ et $\var{X} = 3
    \cdot 100^2/((3 - 1)^2 (3 - 2)) = \nombre{7500}$. Si le seuil de
    crédibilité complète dans le cas Poisson composée est
    $\nombre{2500}$ quand $k = 0,05$, alors
    \begin{displaymath}
      \zeta_{\varepsilon/2}
      = 0,05 \sqrt{\frac{2500}{1 + 7500/50^2}}
      = 1,25.
    \end{displaymath}
    Dans le cas binomiale négative composée, le seuil de crédibilité
    complète est plutôt donné par
    \begin{displaymath}
      r \geq
      \left(
        \frac{\zeta_{\varepsilon/2}}{k}
      \right)^2
      \left(
        \frac{1}{1 - \theta} +
        \frac{\theta}{1-\theta} \frac{\var{X}}{\esp{X}^2}
      \right).
    \end{displaymath}
    Avec $\var{X}/\esp{X}^2 = 3$, $\theta = 0,5$, $r = \nombre{2500}$
    et $\zeta_{\varepsilon/2} = 1,25$, nous obtenons $k = 0,0559$, d'où la
    largeur de l'intervalle de confiance autour de la moyenne est $2 k
    \esp{S} = 2 (0,559) (\nombre{2500}) (50) = \nombre{13975}$.
  \end{sol}
\end{exercice}

\begin{exercice}
  On vous donne les renseignements suivants:
  \begin{enumerate}[(i)]
  \item $S = \sum_{j=1}^{N} X_j$ et les variables aléatoires
    $X_j$ sont mutuellement indépendantes et indépendantes de $N$.
  \item $X_j \sim \text{Pareto}(3, 3)$
  \item $N \sim \text{Binomiale négative}(r, 1/3)$.
  \end{enumerate}
  Calculer la valeur minimale de $r$ pour accorder une crédibilité
  complète d'ordre $(0,05,\, 0,90)$ à $S$. Utiliser l'approximation
  normale.
  \begin{rep}
    $r \geq 3247,23$
  \end{rep}
  \begin{sol}
    Nous avons un cas binomiale négative composée. À l'aide de la
    formule développée à l'\autoref{ex:stabilite:rBNC}, nous obtenons
    directement que
    \begin{align*}
      r
      &\geq
      \left(
        \frac{\zeta_{\varepsilon/2}}{k}
      \right)^2
      \left(
        \frac{1}{1 - \theta} +
        \frac{\theta}{1-\theta} \frac{\var{X}}{\esp{X}^2}
      \right) \\
      &=
      \left(
        \frac{1,645}{0,05}
      \right)^2
      \left(
        \frac{3}{2} + \frac{1}{2} \cdot 3
      \right) \\
      &= \nombre{3247,23}.
    \end{align*}
  \end{sol}
\end{exercice}

%%%
%%% CRÉDIBILITÉ PARTIELLE
%%%

\begin{exercice}
  Soit $S$ la variable aléatoire du montant total des sinistres pour
  un portefeuille d'assurance. Sachant que $S \sim \text{Poisson
    composée}(\lambda, F_X(\cdot))$, trouver le seuil de crédibilité
  complète en termes du nombre espéré de sinistres selon les
  hypothèses de distribution suivantes.
  \begin{enumerate}
  \item $\Pr[X = 1] = 1$ (dégénérée en $1$), $k = 0,05$ et $p = 0,90$.
  \item $X \sim \text{Exponentielle}(2)$, $k = 0,04$, $p = 0,95$.
  \item Si le seuil de crédibilité complète selon les conditions en b)
    n'est pas atteint, quel facteur de crédibilité partielle
    accorderait-on à $\bar{S}$ pour une période d'expérience (en
    fonction de $\lambda$)?
  \end{enumerate}
  \begin{rep}
    \begin{inparaenum}
    \item $\nombre{1082}$
    \item $\nombre{4802}$
    \item $\sqrt{\lambda/\nombre{4802}}$
    \end{inparaenum}
  \end{rep}
  \begin{sol}
    \begin{enumerate}
    \item Avec $X \equiv 1$, $S$ est une Poisson pure et le seuil de
      crédibilité complète est $\lambda \geq
      (\zeta_{\varepsilon/2}/k)^2 = (1,645/0,05)^2 = 1082,41$.
    \item Dans le cas Poisson composée avec $\CV(X) = 1$, le
      seuil de crédibilité complète est $\lambda \geq
      (\zeta_{\varepsilon/2}/k)^2 (1 + \CV(X)^2) = 2
      (1,96/0,04)^2 = \nombre{4802}$.
    \item Sans plus de précisions sur la formule de crédibilité
      partielle à utiliser, nous allons utiliser sur la formule de la
      racine carrée. Pour une seule période d'expérience, nous avons
      $\bar{S} = S$, donc la crédibilité partielle $z$ serait
      \begin{displaymath}
        z = \min
        \left\{
          \sqrt{\frac{\lambda}{\nombre{4802}}}, 1
        \right\}.
      \end{displaymath}
    \end{enumerate}
  \end{sol}
\end{exercice}

\begin{exercice}
  L'actuaire de la compagnie ABC croit qu'il faudrait \nombre{3000}
  sinistres pour accorder une crédibilité complète à un assuré d'un
  groupe si la sévérité est constante.  Après une étude, il se rend
  compte que la sévérité suit plutôt une loi de Pareto de moyenne
  $\nombre{1000}$ avec $\alpha = 3$. Si le nombre de sinistres obéit à
  une loi de Poisson, combien de sinistres doit-on avoir observé si
  l'on a accordé, après une période d'expérience, une crédibilité de
  $0,5$ à l'expérience individuelle de l'assuré?
  \begin{rep}
    \nombre{3000} sinistres
  \end{rep}
  \begin{sol}
    Nous avons un modèle Poisson composée pour le montant total des
    sinistres. Le seuil de crédibilité complète de \nombre{3000} est
    atteint lorsque $\CV(X) = 0$, donc
    $(\zeta_{\varepsilon/2}/k)^2 = \nombre{3000}$. Si nous avons plutôt
    $\CV(X)^2 = \alpha/(\alpha - 2) = 3$, alors le seuil de
    crédibilité complète est $\lambda_0 = \nombre{3000}(1 + 3) =
    \nombre{12000}$. Enfin, si l'on a accordé à l'assuré une
    crédibilité partielle de $z = \sqrt{\lambda/\nombre{12000}} = 0,5$,
    alors c'est que l'assuré a eu $\lambda = \nombre{3000}$
    sinistres.
  \end{sol}
\end{exercice}

\begin{exercice}
  On vous donne les informations suivantes:
  \begin{enumerate}[(i)]
  \item Le nombre de sinistres suit une loi de Poisson.
  \item Le montant des sinistres a une distribution log-normale avec un
    coefficient de variation de 3.
  \item Le nombre et les montants de sinistres sont indépendants.
  \item Le nombre de sinistres la première année fut de \nombre{1000}.
  \item Le montant total des sinistres la première année fut de
    $6,75$~millions.
  \item La prime collective pour la seconde année est de
    $5,00$~millions.
  \item Le volume du contrat est le même pour la première et la
    seconde année.
  \item Le niveau de crédibilité complète assure que le montant total
    des sinistres sera à $5$~\% de la moyenne $95$~\% du temps.
  \end{enumerate}
  Déterminer la prime de crédibilité (partielle) selon l'approche de
  crédibilité de stabilité. Calculer le niveau de crédibilité complète
  sur le nombre d'années d'expérience et utiliser la formule de la
  racine carrée pour calculer le facteur de crédibilité.
  \begin{rep}
    $5,45$
  \end{rep}
  \begin{sol}
    Il faut déduire de l'énoncé que la crédibilité complète est donnée
    en années d'expérience et que $\zeta_{\varepsilon/2} = 1,96$, $k =
    0,05$, $\CV(X) = 3$ et $\lambda = \nombre{1000}$. Puisque
    nous avons un modèle Poisson composée, le niveau de crédibilité
    complète $n_0$ est
    \begin{displaymath}
      n_0 =
      \frac{1}{\nombre{1000}}
      \left( \frac{1,96}{0,05} \right)^2
      (1 + (3)^2) =
      15,3664.
    \end{displaymath}
    Après $n = 1$ année d'expérience, le facteur de crédibilité est
    \begin{equation*}
      z = \sqrt{1/15,3664} = 0,2551.
    \end{equation*}
    Puisque $\bar{S} = S_1 = 6,75$ et $m = 5,00$, la prime de
    crédibilité partielle est
    \begin{displaymath}
      P = 0,2551 (6,75) + (1 - 0,2551) (5) = 5,45.
    \end{displaymath}
  \end{sol}
\end{exercice}

\begin{exercice}
  On vous dit que $S \sim \text{Poisson composée}(\lambda,
  F_X(\cdot))$. La fréquence annuelle espérée des sinistres est
  évaluée à $0,035$. La grandeur minimale du portefeuille de
  l'assureur pour accorder une crédibilité complète à l'expérience est
  $\nombre{103500}$. Pour accorder une crédibilité de $0,67$ à
  l'expérience d'une période, quelle doit être la grandeur minimale du
  portefeuille?
  \begin{rep}
    $\nombre{46462}$
  \end{rep}
  \begin{sol}
    Soit $\lambda$ la grandeur du portefeuille. Le seuil de
    crédibilité complète est $\lambda_0 = \nombre{103500}$. En
    utilisant la formule de la racine carrée, on a donc
    \begin{displaymath}
      z = 0,67 = \sqrt{\frac{\lambda}{\nombre{103500}}},
    \end{displaymath}
    d'où $\lambda = \nombre{46462}$.
  \end{sol}
\end{exercice}

\begin{exercice}
  Les contrats d'une compagnie d'assurance pour un certain type de
  produit ont les caractéristiques ci-dessous.
  \begin{center}
    \begin{tabular}{lcc}
      \toprule
      & Nombre de sinistres & Montant des sinistres \\
      \midrule
      Espérance & 10 & \nombre{5000} \\
      Variance  & 10 & \nombre{6250000} \\
      \bottomrule
    \end{tabular}
  \end{center}
  Une crédibilité complète est accordée à l'expérience d'un contrat
  après $n$ années si celle-ci se concentre dans un intervalle de
  $10$~\% autour de sa moyenne avec probabilité de $90$~\%. Déterminer
  après combien d'années d'expérience le facteur de crédibilité
  partielle sera de $0,54$ selon chacune des formules de crédibilité
  partielle ci-dessous.
  \begin{enumerate}
  \item $z = \left( \dfrac{n}{n_0} \right)^{1/2}$
  \item $z = \left( \dfrac{n}{n_0} \right)^{2/3}$
  \end{enumerate}
  \begin{rep}
    \begin{inparaenum}
    \item $9,86$ années
    \item $13,42$ années
    \end{inparaenum}
  \end{rep}
  \begin{sol}
    Nous avons $\esp{S} = \esp{N} \esp{X} = 50,000$ et
    $\var{S} = \var{N} \esp{X}^2 + \var{X} \esp{N} =
    \nombre{312500000}$. Avec la formule générale
    \eqref{eq:stabilite:complete_comppois}, le seuil de crédibilité
    complète est
    \begin{align*}
      n_0
      &= \left( \frac{\zeta_{\varepsilon/2}}{k} \right)^2
      \frac{\var{S}}{\esp{S}} \\
      &= 270,6025 (0,125) \\
      &= 33,8253.
    \end{align*}
    \begin{enumerate}
    \item Avec la formule de la racine carrée,
      \begin{displaymath}
        0,54 = \sqrt{\frac{n}{33,8253}} \Rightarrow n = 9,86.
      \end{displaymath}
    \item Avec la formule «puissance 2/3»,
      \begin{displaymath}
        0,54 = \left( \frac{n}{33,8253} \right)^{2/3} \Rightarrow n = 13,42.
      \end{displaymath}
      Le temps requis pour atteindre la pleine crédibilité est donc
      plus élevé avec cette formule.
    \end{enumerate}
  \end{sol}
\end{exercice}

\Closesolutionfile{solutions}
\Closesolutionfile{reponses}

%%% Local Variables:
%%% mode: latex
%%% TeX-engine: xetex
%%% TeX-master: "theorie-credibilite-avec-r"
%%% coding: utf-8
%%% End:
