%%% Copyright (C) 2018 Vincent Goulet
%%%
%%% Ce fichier fait partie du projet
%%% «Théorie de la crédibilité avec R»
%%% http://github.com/vigou3/theorie-credibilite-avec-r
%%%
%%% Cette création est mise à disposition selon le contrat
%%% Attribution-Partage dans les mêmes conditions 4.0
%%% International de Creative Commons.
%%% http://creativecommons.org/licenses/by-sa/4.0/

\chapter*{Introduction}
\addcontentsline{toc}{chapter}{Introduction}
\markboth{Introduction}{Introduction}

«Les mathématiques de l'hétérogénéité.» C'est parfois ainsi que l'on
décrit la théorie de la crédibilité, pierre angulaire des
mathématiques de l'assurance IARD. Cet ouvrage offre un traitement
rigoureux et exhaustif des modèles de base de la crédibilité, soit la
crédibilité de stabilité (\emph{limited fluctuations}), la
tarification basée sur l'expérience purement bayésienne et les modèles
classiques de Bühlmann et de Bühlmann--Straub.

Le paquetage \pkg{actuar} \citep{actuar} pour l'environnement
statistique R \citep{R} permet d'effectuer les calculs relatifs aux
modèles de crédibilité abordés dans l'ouvrage. Nous expliquons comment
utiliser la fonction \code{cm} du paquetage par le biais de code
informatique distribué avec le document et reproduit à la fin des
chapitres \ref*{chap:bayesienne}, \ref*{chap:buhlmann} et
\ref*{chap:buhlmann-straub}.

L'ouvrage intègre également le recueil d'exercices et de solutions
de \citet{Cossette:credibilite:2008}. Tel que mentionné en
introduction de ce document, la collection d'exercices est le fruit de
la mise en commun d'exercices colligés au fil du temps pour des cours
de théorie de la crédibilité à l'Université Laval et à l'Université
Concordia. Certains exercices ont été rédigés par les Professeurs
François Dufresne et Jacques Rioux, entre autres. Quelques exercices
proviennent également d'anciens examens de la Society of Actuaries et
de la Casualty Actuarial Society.

Le premier chapitre, tiré de \cite{Goulet:masters}, trace l'historique et
l'évolution de la théorie de la crédibilité, de ses origines jusqu'au
début des années 1990. Ce chapitre ne comporte pas d'exercices.

L'\autoref{chap:estimation-bayesienne} offre un sommaire des
principaux principes et résultats en estimation bayésienne. Un tableau
synoptique des principaux résultats de crédibilité exacte se trouve à
l'annexe \ref{chap:formules}. L'annexe \ref{chap:distributions},
présente la paramétrisation des lois de probabilité utilisée dans les
exercices. En cas de doute, le lecteur est invité à la consulter. Il y
trouvera également l'espérance, la variance et la fonction génératrice
des moments (lorsqu'elle existe) des lois de probabilités rencontrées
dans ce document.

%%% Local Variables:
%%% TeX-master: "theorie-credibilite-avec-r"
%%% TeX-engine: xetex
%%% coding: utf-8
%%% End:
