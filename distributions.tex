\chapter{Paramétrisation des lois de probabilité}
\label{chap:distributions}

Cette annexe précise la paramétrisation des lois de probabilité
utilisée dans les énoncés des exercices des chapitres
\ref{chap:stabilite} à \ref{chap:buhlmann-straub} ainsi que dans le
tableau des formules de crédibilité exacte de l'annexe
\ref{chap:formules}; la racine des noms de fonctions pour calculer les
caractéristiques des lois dans R; l'espérance, la variance et la
fonction génératrice des moments (lorsqu'elle existe) des différentes
lois.

La prise en charge de certaines lois dans R requiert le paquetage
\pkg{actuar} \citep{actuar}.


%% Listes à puce compactes et sans puce, justement.
\begingroup
\setlist[itemize]{label={},leftmargin=0pt,align=left,nosep}

\section{Distributions discrètes}

\subsection{Binomiale}
\label{distributions:binomiale}

\begin{itemize}
\item Racine: \code{binom}
\item Paramètres: \code{size} ($n = 0, 1, 2, \dots$),
  \code{prob} ($0 \leq \theta \leq 1$)
\end{itemize}
\begin{itemize}
\item Cas spécial: Bernoulli($\theta)$ lorsque $n = 1$.
\end{itemize}
\begin{align*}
  \Pr[X = x]
  &= \binom{n}{x} \theta^x (1 - \theta)^{n - x}, \quad
  x = 0, 1, \dots \\
  \esp{X}
  &= n \theta \\
  \var{X}
  &= n \theta (1 - \theta) \\
  M(t)
  &= (1 - \theta + \theta e^t)^n
\end{align*}

\subsection{Binomiale négative}
\label{distributions:binomialeneg}

\begin{itemize}
\item Racine: \code{nbinom}
\item Paramètres: \code{size} ($r \geq 0$),
  \code{prob} ($0 < \theta \leq 1$),
  \code{mu} ($r(1 - \theta)/\theta$)
\end{itemize}
\begin{itemize}
\item Cas spécial: Géométrique($\theta)$ lorsque $r = 1$.
\end{itemize}
\begin{align*}
  \Pr[X = x]
  &= \binom{x + r - 1}{r - 1} \theta^r (1 - \theta)^x, \quad
  x = 0, 1, \dots \\
  \esp{X}
  &= \frac{r (1 - \theta)}{\theta} \\
  \var{X}
  &= \frac{r (1 - \theta)}{\theta^2} \\
  M(t)
  &= \left( \frac{\theta}{1 - (1 - \theta)e^t} \right)^r
\end{align*}

\subsection{Poisson}
\label{distributions:poisson}

\begin{itemize}
\item Racine: \code{pois}
\item Paramètre: \code{lambda} ($\lambda \geq 0$)
\end{itemize}
\begin{align*}
  \Pr[X = x]
  &= \frac{\lambda^x e^{-\lambda}}{x!}, \quad x = 0, 1, \dots \\
  \esp{X}
  &= \lambda \\
  \var{X}
  &= \lambda \\
  M(t)
  &= e^{\lambda (e^t - 1)}
\end{align*}


\section{Distributions continues}

\subsection{Bêta}
\label{distributions:beta}

\begin{itemize}
\item Racine: \code{beta}
\item Paramètres: \code{shape1} ($a > 0$),
      \code{shape2} ($b > 0$)
\end{itemize}
\begin{itemize}
\item Cas spécial: Uniforme$(0, 1)$ lorsque $a = b = 1$.
\end{itemize}
\begin{align*}
  f(x)
  &= \frac{\Gamma(a + b)}{\Gamma(a) \Gamma(b)}\,
  x^{a - 1} (1 - x)^{b - 1}, \quad 0 < x < 1 \\
  \esp{X}
  &= \frac{a}{a + b} \\
  \var{X}
  &= \frac{a b}{(a + b)^2 (a + b + 1)}
\end{align*}


\subsection{Gamma}
\label{distributions:gamma}

\begin{itemize}
\item Racine: \code{gamma}
\item Paramètres: \code{shape} ($\alpha > 0$),
      \code{rate}   ($\lambda = 1/\theta$),
      \code{scale}  ($\theta > 0$)
\end{itemize}
\begin{itemize}
\item Cas spéciaux: Exponentielle$(\lambda)$ lorsque $\alpha = 1$,
  $\chi^2(r)$ lorsque $\alpha = r/2$ et $\lambda = 1/2$.
\end{itemize}
\begin{align*}
  f(x)
  &= \frac{\lambda^\alpha}{\Gamma(\alpha)}\, x^{\alpha - 1} e^{-\lambda x},
  \quad x > 0 \\
  \esp{X}
  &= \frac{\alpha}{\lambda} \\
  \var{X}
  &= \frac{\alpha}{\lambda^2} \\
  M(t)
  &= \left( \frac{\lambda}{\lambda - t} \right)^\alpha
\end{align*}


\subsection{Normale}
\label{distributions:normale}
\begin{itemize}
\item Racine: \code{norm}
\item Paramètres: \code{mean} ($-\infty < \mu < \infty$), \code{sd}
  ($\sigma > 0$)
\end{itemize}
\begin{align*}
  f(x)
  &= \frac{1}{\sqrt{2\pi} \sigma}\,
  \exp\left\{ -\frac{(x - \mu)^2}{2\sigma^2} \right\}, \quad
  -\infty < x < \infty \\
  \esp{X}
  &= \mu \\
  \var{X}
  &= \sigma^2 \\
  M(t)
  &= e^{\mu t + \sigma^2 t^2/2}
\end{align*}

\subsection{Log-normale}
\label{distributions:log-normale}

\begin{itemize}
\item Racine: \code{lnorm}
\item Paramètres: \code{meanlog} ($-\infty < \mu < \infty$), \code{sdlog}
  ($\sigma > 0$)
\end{itemize}
\begin{align*}
  f(x)
  &= \frac{1}{\sqrt{2\pi} \sigma} \frac{1}{x}\,
  \exp\left\{ -\frac{(\ln x - \mu)^2}{2\sigma^2} \right\}, \quad
  x > 0 \\
  \esp{X}
  &= e^{\mu + \sigma^2/2} \\
  \var{X}
  &= e^{2\mu + \sigma^2}(e^{\sigma^2} - 1) \\
\end{align*}

\subsection{Pareto}
\label{distributions:pareto}

\begin{itemize}
\item Racine: \code{pareto}, \code{pareto2} (\pkg{actuar})
\item Paramètres: \code{shape} ($\alpha > 0$),
      \code{scale}  ($\theta > 0$)
\end{itemize}
\begin{align*}
  f(x) &=
  \frac{\alpha \theta^\alpha}{(x + \theta)^{\alpha+1}},
  \quad x > 0 \displaybreak[0] \\
  \esp{X}
  &= \frac{\theta}{\alpha - 1}, \quad \alpha > 1 \displaybreak[0] \\
  \var{X}
  &= \frac{\alpha \theta^2}{(\alpha - 1)^2 (\alpha - 2)}, \quad
  \alpha > 2
\end{align*}

\subsection{Pareto généralisée}
\label{distributions:paretogen}

\begin{itemize}
\item Racine: \code{genpareto} (\pkg{actuar})
\item Paramètres: \code{shape1} ($\alpha > 0$),
      \code{shape2} ($\tau > 0$),
      \code{rate}   ($\lambda = 1/\theta$),
      \code{scale}  ($\theta > 0$)
\end{itemize}
\begin{itemize}
\item Cas spécial: Pareto$(\alpha, \lambda)$ lorsque $\tau = 1$.
\end{itemize}
\begin{align*}
  f(x) &=
  \frac{\Gamma(\alpha + \tau)}{\Gamma(\alpha) \Gamma(\tau)}\,
  \frac{\theta^\alpha x^{\tau - 1}}{(x + \theta)^{\alpha+\tau}},
  \quad x > 0 \\
  \esp{X}
  &= \frac{\theta \tau}{\alpha - 1}, \quad \alpha > 1 \\
  \var{X}
  &= \frac{\theta^2 \tau (\alpha + \tau - 1)}{(\alpha - 1)^2 (\alpha
    - 2)}, \quad \alpha > 2
\end{align*}


\section{Distributions composées}

Soit $S = X_1 + \dots + X_N$, où $X_1, \dots, X_n$ sont des variables
aléatoires mutuellement indépendantes, identiquement distribuées et
toutes indépendantes de $N$. On a toujours
\begin{align*}
  \esp{S} &= \esp{N} \esp{X} \\
  \var{S} &= \var{N} \esp{X}^2 + \esp{N} \var{X}.
\end{align*}

\subsection{Binomiale composée}
\label{distributions:binomialecomposée}

Distribution de $S$ lorsque $N \sim \text{Binomiale}(n, \theta)$ et
$\Pr[X \leq x] = F_X(x)$.
\begin{align*}
  \esp{S} &= n \theta \esp{X} \\
  \var{S} &= n\theta(1 - \theta) \esp{X}^2 + n \theta \var{X}
\end{align*}

\subsection{Binomiale négative composée}
\label{distributions:binomialenegativecomposée}

Distribution de $S$ lorsque $N \sim \text{Binomiale négative}(r,
\theta)$ et $\Pr[X \leq x] = F_X(x)$.
\begin{align*}
  \esp{S} &= \frac{r(1 - \theta)}{\theta}\, \esp{X} \\
  \var{S} &= \frac{r(1 - \theta)}{\theta^2}\, \esp{X}^2 + \frac{r(1 -
    \theta)}{\theta}\, \var{X}
\end{align*}

\subsection{Poisson composée}
\label{distributions:poissoncomposée}

Distribution de $S$ lorsque $N \sim \text{Poisson}(\lambda)$ et $\Pr[X
\leq x] = F_X(x)$.
\begin{align*}
  \esp{S} &= \lambda \esp{X} \\
  \var{S} &= \lambda \esp{X^2}
\end{align*}

\endgroup

%%% Local Variables:
%%% mode: latex
%%% TeX-engine: xetex
%%% TeX-master: "theorie-credibilite-avec-r"
%%% coding: utf-8
%%% End:
