\chapter{Estimation bayésienne}
\label{chap:estimation-bayesienne}

L'estimation ponctuelle par l'approche bayésienne diffère passablement
des approches classiques comme la méthode des moments ou la méthode du
maximum de vraisemblance. Cette annexe propose un résumé de la
philosophie et l'estimation bayésienne ainsi que les principaux
résultats.

\section{Cas continu}
\label{sec:estimation-bayesienne:continu}

Supposons que nous souhaitons estimer le paramètre inconnu $\theta$
d'une distribution continue avec fonction de densité de probabilité
$f(x;\theta)$ (une loi normale avec moyenne inconnue $\theta$ et
variance connue $\sigma^2$, par exemple) à partir d'un échantillon
aléatoire $X_1, \dots, X_n$.

Les statisticiens classiques développeront des estimateurs à partir
d'un critère objectif quelconque: absence de biais, maximum de
vraisemblance, etc. Vous remarquerez que l'analyste ne pose aucune
hypothèse à priori sur la valeur de $\theta$. On «laisse parler les
données».

Dans l'approche bayésienne, l'opinion à priori d'une personne sur la
valeur du paramètre $\theta$ est prise en compte dans l'estimation de
ce dernier. Le truc consiste à considérer le paramètre comme une
réalisation d'une variable aléatoire $\Theta$ avec densité $u(\theta)$
qui représente l'opinion de la l'analyste quant à la valeur du
paramètre. Au fur et à mesure que les données de l'échantillon
aléatoire (l'information) s'accumulent, l'analyste améliore son
opinion et, par conséquent, révise la distribution de la variable
aléatoire $\Theta$ en calculant sa densité à postériori
$u(\theta|x_1, \dots, x_n)$ à l'aide de la régle de Bayes:
\begin{align*}
  u(\theta|x_1, \dots, x_n)
  &= \frac{f(x_1, \dots, x_n, \theta)}{f(x_1, \dots, x_n)} \\
  &= \frac{f(x_1, \dots, x_n|\theta) u(\theta)}{f(x_1, \dots, x_n)}.
\end{align*}
Par la loi des probabilités totales,
\begin{equation*}
  f(x_1, \dots, x_n) =
  \int_{-\infty}^\infty f(x_1, \dots, x_n|\theta) u(\theta)\,d\theta,
\end{equation*}
et donc
\begin{equation*}
  u(\theta|x_1, \dots, x_n)
  = \frac{f(x_1, \dots, x_n|\theta) u(\theta)}
      {\int_{-\infty}^\infty f(x_1, \dots, x_n|\theta) u(\theta)\, d\theta}.
\end{equation*}
Le dénominateur du côté droit de l'expression ci-dessus n'est qu'une
constante de normalisation qui est souvent omise dans les calculs.

Pour obtenir un estimateur ponctuel
$\hat{\theta} = g(X_1, \dots, X_n)$ du paramètre $\theta$, nous
minimisons l'espérance à postériori d'une \emph{fonction de perte}. La
fonction de perte la plus fréquemment employée est l'erreur
quadratique
\begin{equation*}
  L(\hat{\theta}, \theta) = (\hat{\theta} - \theta)^2.
\end{equation*}
Or, l'estimateur bayésien qui minimise
\begin{equation*}
  \esp{L(\hat{\theta}, \theta)|X_1, \dots, X_n} =
  \esp{(\hat{\theta} - \theta)^2|X_1, \dots, X_n}
\end{equation*}
est
\begin{align*}
  \hat{\theta}
  &= \esp{\Theta|X_1, \dots, X_n} \\
  &= \int_{-\infty}^\infty \theta\, u(\theta|x_1, \dots, x_n)\, d\theta,
\end{align*}
soit l'espérance de $\Theta$ calculée par rapport à la distribution à
postériori.


\section{Cas discret}
\label{sec:estimation-bayesienne:discret}

Les idées de la section précédente demeurent exactement les mêmes dans
le cas discret; seule la notation change légèrement. Pour simplifier
la notation, posons $\mat{X} = (X_1, \dots, X_n)$ et
$\mat{x} = (x_1, \dots, x_n)$.

Si la variable aléatoire $\Theta$ ne prend que des valeurs discrètes,
la distribution à priori est exprimée sous forme d'une fonction de
masse de probabilité $\Pr[\Theta = \theta]$. Nous pouvons toujours
calculer la fonction de masse de probabilité conjointe de
$X_1, \dots, X_n$ par la loi des probabilités totales:
\begin{equation*}
  \Pr[\mat{X} = \mat{x}]
  = \sum_{\theta=-\infty}^\infty
  \Pr[\mat{X} = \mat{x}|\Theta = \theta] \Pr[\Theta = \theta].
\end{equation*}
La règle de Bayes permet de calculer la distribution à postériori de
$\Theta$:
\begin{align*}
  \Pr[\Theta = \theta|\mat{X} = \mat{x}]
  &= \frac{\Pr[\mat{X} = \mat{x}|\Theta = \theta] \Pr[\Theta =
    \theta]}{\Pr[\mat{X} = \mat{x}]} \\
  &= \frac{\Pr[\mat{X} = \mat{x}|\Theta = \theta] \Pr[\Theta =
    \theta]}{\sum_{\theta=-\infty}^\infty \Pr[\mat{X} =
    \mat{x}|\Theta = \theta]
    \Pr[\Theta = \theta]}.
\end{align*}
Enfin, l'estimateur bayésien minimisant l'erreur quadratique moyenne
demeure inchangé:
\begin{align*}
  \hat{\theta}
  &= \Esp{\Theta|X_1, \dots, X_n} \\
  &= \sum_{\theta=-\infty}^\infty \theta\, \Pr[\Theta =
    \theta|\mat{X} = \mat{x}].
\end{align*}


\section{Cas mixtes}
\label{sec:estimation-bayesienne:mixte}

Il est simple de dériver les formules des cas mixtes, les plus
courants étant ceux où la distribution de $X|\Theta$ est discrète et
celle de $\Theta$, continue.

%%% Local Variables:
%%% TeX-master: "theorie-credibilite-avec-r"
%%% TeX-engine: xetex
%%% coding: utf-8
%%% End:
